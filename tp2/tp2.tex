\documentclass[a4paper]{article}
\usepackage[a4paper,left=2.5cm,right=2cm,top=2.5cm,bottom=2.5cm]{geometry}
\usepackage{palatino}
\usepackage[colorlinks=true,linkcolor=blue,citecolor=blue]{hyperref}
\usepackage{graphicx}
\usepackage{tp2}
\usepackage{subcaption}
\usepackage{adjustbox}
\usepackage{color}

\definecolor{red}{RGB}{255,  0,  0}
\definecolor{blue}{RGB}{0,0,255}
\def\red{\color{red}}
\def\blue{\color{blue}}

%================= lhs2tex=====================================================%
%% ODER: format ==         = "\mathrel{==}"
%% ODER: format /=         = "\neq "
%
%
\makeatletter
\@ifundefined{lhs2tex.lhs2tex.sty.read}%
  {\@namedef{lhs2tex.lhs2tex.sty.read}{}%
   \newcommand\SkipToFmtEnd{}%
   \newcommand\EndFmtInput{}%
   \long\def\SkipToFmtEnd#1\EndFmtInput{}%
  }\SkipToFmtEnd

\newcommand\ReadOnlyOnce[1]{\@ifundefined{#1}{\@namedef{#1}{}}\SkipToFmtEnd}
\usepackage{amstext}
\usepackage{amssymb}
\usepackage{stmaryrd}
\DeclareFontFamily{OT1}{cmtex}{}
\DeclareFontShape{OT1}{cmtex}{m}{n}
  {<5><6><7><8>cmtex8
   <9>cmtex9
   <10><10.95><12><14.4><17.28><20.74><24.88>cmtex10}{}
\DeclareFontShape{OT1}{cmtex}{m}{it}
  {<-> ssub * cmtt/m/it}{}
\newcommand{\texfamily}{\fontfamily{cmtex}\selectfont}
\DeclareFontShape{OT1}{cmtt}{bx}{n}
  {<5><6><7><8>cmtt8
   <9>cmbtt9
   <10><10.95><12><14.4><17.28><20.74><24.88>cmbtt10}{}
\DeclareFontShape{OT1}{cmtex}{bx}{n}
  {<-> ssub * cmtt/bx/n}{}
\newcommand{\tex}[1]{\text{\texfamily#1}}	% NEU

\newcommand{\Sp}{\hskip.33334em\relax}


\newcommand{\Conid}[1]{\mathit{#1}}
\newcommand{\Varid}[1]{\mathit{#1}}
\newcommand{\anonymous}{\kern0.06em \vbox{\hrule\@width.5em}}
\newcommand{\plus}{\mathbin{+\!\!\!+}}
\newcommand{\bind}{\mathbin{>\!\!\!>\mkern-6.7mu=}}
\newcommand{\rbind}{\mathbin{=\mkern-6.7mu<\!\!\!<}}% suggested by Neil Mitchell
\newcommand{\sequ}{\mathbin{>\!\!\!>}}
\renewcommand{\leq}{\leqslant}
\renewcommand{\geq}{\geqslant}
\usepackage{polytable}

%mathindent has to be defined
\@ifundefined{mathindent}%
  {\newdimen\mathindent\mathindent\leftmargini}%
  {}%

\def\resethooks{%
  \global\let\SaveRestoreHook\empty
  \global\let\ColumnHook\empty}
\newcommand*{\savecolumns}[1][default]%
  {\g@addto@macro\SaveRestoreHook{\savecolumns[#1]}}
\newcommand*{\restorecolumns}[1][default]%
  {\g@addto@macro\SaveRestoreHook{\restorecolumns[#1]}}
\newcommand*{\aligncolumn}[2]%
  {\g@addto@macro\ColumnHook{\column{#1}{#2}}}

\resethooks

\newcommand{\onelinecommentchars}{\quad-{}- }
\newcommand{\commentbeginchars}{\enskip\{-}
\newcommand{\commentendchars}{-\}\enskip}

\newcommand{\visiblecomments}{%
  \let\onelinecomment=\onelinecommentchars
  \let\commentbegin=\commentbeginchars
  \let\commentend=\commentendchars}

\newcommand{\invisiblecomments}{%
  \let\onelinecomment=\empty
  \let\commentbegin=\empty
  \let\commentend=\empty}

\visiblecomments

\newlength{\blanklineskip}
\setlength{\blanklineskip}{0.66084ex}

\newcommand{\hsindent}[1]{\quad}% default is fixed indentation
\let\hspre\empty
\let\hspost\empty
\newcommand{\NB}{\textbf{NB}}
\newcommand{\Todo}[1]{$\langle$\textbf{To do:}~#1$\rangle$}

\EndFmtInput
\makeatother
%
%
%
%
%
%
% This package provides two environments suitable to take the place
% of hscode, called "plainhscode" and "arrayhscode". 
%
% The plain environment surrounds each code block by vertical space,
% and it uses \abovedisplayskip and \belowdisplayskip to get spacing
% similar to formulas. Note that if these dimensions are changed,
% the spacing around displayed math formulas changes as well.
% All code is indented using \leftskip.
%
% Changed 19.08.2004 to reflect changes in colorcode. Should work with
% CodeGroup.sty.
%
\ReadOnlyOnce{polycode.fmt}%
\makeatletter

\newcommand{\hsnewpar}[1]%
  {{\parskip=0pt\parindent=0pt\par\vskip #1\noindent}}

% can be used, for instance, to redefine the code size, by setting the
% command to \small or something alike
\newcommand{\hscodestyle}{}

% The command \sethscode can be used to switch the code formatting
% behaviour by mapping the hscode environment in the subst directive
% to a new LaTeX environment.

\newcommand{\sethscode}[1]%
  {\expandafter\let\expandafter\hscode\csname #1\endcsname
   \expandafter\let\expandafter\endhscode\csname end#1\endcsname}

% "compatibility" mode restores the non-polycode.fmt layout.

\newenvironment{compathscode}%
  {\par\noindent
   \advance\leftskip\mathindent
   \hscodestyle
   \let\\=\@normalcr
   \let\hspre\(\let\hspost\)%
   \pboxed}%
  {\endpboxed\)%
   \par\noindent
   \ignorespacesafterend}

\newcommand{\compaths}{\sethscode{compathscode}}

% "plain" mode is the proposed default.
% It should now work with \centering.
% This required some changes. The old version
% is still available for reference as oldplainhscode.

\newenvironment{plainhscode}%
  {\hsnewpar\abovedisplayskip
   \advance\leftskip\mathindent
   \hscodestyle
   \let\hspre\(\let\hspost\)%
   \pboxed}%
  {\endpboxed%
   \hsnewpar\belowdisplayskip
   \ignorespacesafterend}

\newenvironment{oldplainhscode}%
  {\hsnewpar\abovedisplayskip
   \advance\leftskip\mathindent
   \hscodestyle
   \let\\=\@normalcr
   \(\pboxed}%
  {\endpboxed\)%
   \hsnewpar\belowdisplayskip
   \ignorespacesafterend}

% Here, we make plainhscode the default environment.

\newcommand{\plainhs}{\sethscode{plainhscode}}
\newcommand{\oldplainhs}{\sethscode{oldplainhscode}}
\plainhs

% The arrayhscode is like plain, but makes use of polytable's
% parray environment which disallows page breaks in code blocks.

\newenvironment{arrayhscode}%
  {\hsnewpar\abovedisplayskip
   \advance\leftskip\mathindent
   \hscodestyle
   \let\\=\@normalcr
   \(\parray}%
  {\endparray\)%
   \hsnewpar\belowdisplayskip
   \ignorespacesafterend}

\newcommand{\arrayhs}{\sethscode{arrayhscode}}

% The mathhscode environment also makes use of polytable's parray 
% environment. It is supposed to be used only inside math mode 
% (I used it to typeset the type rules in my thesis).

\newenvironment{mathhscode}%
  {\parray}{\endparray}

\newcommand{\mathhs}{\sethscode{mathhscode}}

% texths is similar to mathhs, but works in text mode.

\newenvironment{texthscode}%
  {\(\parray}{\endparray\)}

\newcommand{\texths}{\sethscode{texthscode}}

% The framed environment places code in a framed box.

\def\codeframewidth{\arrayrulewidth}
\RequirePackage{calc}

\newenvironment{framedhscode}%
  {\parskip=\abovedisplayskip\par\noindent
   \hscodestyle
   \arrayrulewidth=\codeframewidth
   \tabular{@{}|p{\linewidth-2\arraycolsep-2\arrayrulewidth-2pt}|@{}}%
   \hline\framedhslinecorrect\\{-1.5ex}%
   \let\endoflinesave=\\
   \let\\=\@normalcr
   \(\pboxed}%
  {\endpboxed\)%
   \framedhslinecorrect\endoflinesave{.5ex}\hline
   \endtabular
   \parskip=\belowdisplayskip\par\noindent
   \ignorespacesafterend}

\newcommand{\framedhslinecorrect}[2]%
  {#1[#2]}

\newcommand{\framedhs}{\sethscode{framedhscode}}

% The inlinehscode environment is an experimental environment
% that can be used to typeset displayed code inline.

\newenvironment{inlinehscode}%
  {\(\def\column##1##2{}%
   \let\>\undefined\let\<\undefined\let\\\undefined
   \newcommand\>[1][]{}\newcommand\<[1][]{}\newcommand\\[1][]{}%
   \def\fromto##1##2##3{##3}%
   \def\nextline{}}{\) }%

\newcommand{\inlinehs}{\sethscode{inlinehscode}}

% The joincode environment is a separate environment that
% can be used to surround and thereby connect multiple code
% blocks.

\newenvironment{joincode}%
  {\let\orighscode=\hscode
   \let\origendhscode=\endhscode
   \def\endhscode{\def\hscode{\endgroup\def\@currenvir{hscode}\\}\begingroup}
   %\let\SaveRestoreHook=\empty
   %\let\ColumnHook=\empty
   %\let\resethooks=\empty
   \orighscode\def\hscode{\endgroup\def\@currenvir{hscode}}}%
  {\origendhscode
   \global\let\hscode=\orighscode
   \global\let\endhscode=\origendhscode}%

\makeatother
\EndFmtInput
%
\newlabel{eq:fokkinga}{{3.93}{110}{The mutual-recursion law}{section.3.17}{}}
\def\plus{\mathbin{\dagger}}
\def\ana#1{\mathopen{[\!(}#1\mathclose{)\!]}}

%---------------------------------------------------------------------------

\begin{document}

\setlength\parindent{0pt}

\title{\bfseries Modelling and Analysis of a Cyber-Physical System \\ {\Large Cyber-Physical Programming --- Practical Assignment 1}}

\author{
    Melânia Pereira \quad\quad Paulo R. Pereira\\\texttt{\{pg47520, pg47554\}@alunos.uminho.pt}
}

\maketitle


The list of adventurers
\begin{hscode}\SaveRestoreHook
\column{B}{@{}>{\hspre}l<{\hspost}@{}}%
\column{E}{@{}>{\hspre}l<{\hspost}@{}}%
\>[B]{}\mathbf{data}\;\Conid{Adventurer}\mathrel{=}P_1 \mid P_2 \mid P_5 \mid P_{10} \;\mathbf{deriving}\;(\Conid{Show},\Conid{Eq}){}\<[E]%
\ColumnHook
\end{hscode}\resethooks
Adventurers + the lantern
\begin{hscode}\SaveRestoreHook
\column{B}{@{}>{\hspre}l<{\hspost}@{}}%
\column{E}{@{}>{\hspre}l<{\hspost}@{}}%
\>[B]{}\mathbf{type}\;\Conid{Objects}\mathrel{=}\Conid{Adventurer}+(){}\<[E]%
\ColumnHook
\end{hscode}\resethooks
The time that each adventurer needs to cross the bridge
\begin{hscode}\SaveRestoreHook
\column{B}{@{}>{\hspre}l<{\hspost}@{}}%
\column{E}{@{}>{\hspre}l<{\hspost}@{}}%
\>[B]{}\Varid{getTimeAdv}\mathbin{::}\Conid{Adventurer}\to \Conid{Int}{}\<[E]%
\\
\>[B]{}\Varid{getTimeAdv}\mathrel{=}\bot {}\<[E]%
\ColumnHook
\end{hscode}\resethooks

The state of the game, i.e. the current position of each adventurer
+ the lantern. The function (const False) represents the initial state
of the game, with all adventurers and the lantern on the left side of
the bridge. Similarly, the function (const True) represents the end
state of the game, with all adventurers and the lantern on the right
side of the bridge.
\begin{hscode}\SaveRestoreHook
\column{B}{@{}>{\hspre}l<{\hspost}@{}}%
\column{3}{@{}>{\hspre}l<{\hspost}@{}}%
\column{21}{@{}>{\hspre}l<{\hspost}@{}}%
\column{34}{@{}>{\hspre}l<{\hspost}@{}}%
\column{E}{@{}>{\hspre}l<{\hspost}@{}}%
\>[B]{}\mathbf{type}\;\Conid{State}\mathrel{=}\Conid{Objects}\to \Conid{Bool}{}\<[E]%
\\[\blanklineskip]%
\>[B]{}\mathbf{instance}\;\Conid{Show}\;\Conid{State}\;\mathbf{where}{}\<[E]%
\\
\>[B]{}\hsindent{3}{}\<[3]%
\>[3]{}\Varid{show}\;\Varid{s}\mathrel{=}(\Varid{show}\comp (\mathsf{fmap}\;\Varid{show}))\;[\mskip1.5mu \Varid{s}\;(i_1\;P_1 ),{}\<[E]%
\\
\>[3]{}\hsindent{31}{}\<[34]%
\>[34]{}\Varid{s}\;(i_1\;P_2 ),{}\<[E]%
\\
\>[3]{}\hsindent{31}{}\<[34]%
\>[34]{}\Varid{s}\;(i_1\;P_5 ),{}\<[E]%
\\
\>[3]{}\hsindent{31}{}\<[34]%
\>[34]{}\Varid{s}\;(i_1\;P_{10} ),{}\<[E]%
\\
\>[3]{}\hsindent{31}{}\<[34]%
\>[34]{}\Varid{s}\;(i_2\;())\mskip1.5mu]{}\<[E]%
\\[\blanklineskip]%
\>[B]{}\mathbf{instance}\;\Conid{Eq}\;\Conid{State}\;\mathbf{where}{}\<[E]%
\\
\>[B]{}\hsindent{3}{}\<[3]%
\>[3]{}(\equiv )\;\Varid{s1}\;\Varid{s2}\mathrel{=}\Varid{and}\;[\mskip1.5mu \Varid{s1}\;(i_1\;P_1 )\equiv \Varid{s2}\;(i_1\;P_1 ),{}\<[E]%
\\
\>[3]{}\hsindent{18}{}\<[21]%
\>[21]{}\Varid{s1}\;(i_1\;P_2 )\equiv \Varid{s2}\;(i_1\;P_2 ),{}\<[E]%
\\
\>[3]{}\hsindent{18}{}\<[21]%
\>[21]{}\Varid{s1}\;(i_1\;P_5 )\equiv \Varid{s2}\;(i_1\;P_5 ),{}\<[E]%
\\
\>[3]{}\hsindent{18}{}\<[21]%
\>[21]{}\Varid{s1}\;(i_1\;P_{10} )\equiv \Varid{s2}\;(i_1\;P_{10} ),{}\<[E]%
\\
\>[3]{}\hsindent{18}{}\<[21]%
\>[21]{}\Varid{s1}\;(i_2\;())\equiv \Varid{s2}\;(i_2\;())\mskip1.5mu]{}\<[E]%
\ColumnHook
\end{hscode}\resethooks
The initial state of the game
\begin{hscode}\SaveRestoreHook
\column{B}{@{}>{\hspre}l<{\hspost}@{}}%
\column{E}{@{}>{\hspre}l<{\hspost}@{}}%
\>[B]{}\Varid{gInit}\mathbin{::}\Conid{State}{}\<[E]%
\\
\>[B]{}\Varid{gInit}\mathrel{=}\underline{\Conid{False}}{}\<[E]%
\ColumnHook
\end{hscode}\resethooks

Changes the state of the game for a given object
\begin{hscode}\SaveRestoreHook
\column{B}{@{}>{\hspre}l<{\hspost}@{}}%
\column{E}{@{}>{\hspre}l<{\hspost}@{}}%
\>[B]{}\Varid{changeState}\mathbin{::}\Conid{Objects}\to \Conid{State}\to \Conid{State}{}\<[E]%
\\
\>[B]{}\Varid{changeState}\;\Varid{a}\;\Varid{s}\mathrel{=}\mathbf{let}\;\Varid{v}\mathrel{=}\Varid{s}\;\Varid{a}\;\mathbf{in}\;(\lambda \Varid{x}\to \mathbf{if}\;\Varid{x}\equiv \Varid{a}\;\mathbf{then}\;\neg \;\Varid{v}\;\mathbf{else}\;\Varid{s}\;\Varid{x}){}\<[E]%
\ColumnHook
\end{hscode}\resethooks

Changes the state of the game of a list of objects
\begin{hscode}\SaveRestoreHook
\column{B}{@{}>{\hspre}l<{\hspost}@{}}%
\column{E}{@{}>{\hspre}l<{\hspost}@{}}%
\>[B]{}\Varid{mChangeState}\mathbin{::}[\mskip1.5mu \Conid{Objects}\mskip1.5mu]\to \Conid{State}\to \Conid{State}{}\<[E]%
\\
\>[B]{}\Varid{mChangeState}\;\Varid{os}\;\Varid{s}\mathrel{=}\Varid{foldr}\;\Varid{changeState}\;\Varid{s}\;\Varid{os}{}\<[E]%
\ColumnHook
\end{hscode}\resethooks

For a given state of the game, the function presents all the
possible moves that the adventurers can make.
\begin{hscode}\SaveRestoreHook
\column{B}{@{}>{\hspre}l<{\hspost}@{}}%
\column{E}{@{}>{\hspre}l<{\hspost}@{}}%
\>[B]{}\Varid{allValidPlays}\mathbin{::}\Conid{State}\to \Conid{ListDur}\;\Conid{State}{}\<[E]%
\\
\>[B]{}\Varid{allValidPlays}\mathrel{=}\bot {}\<[E]%
\ColumnHook
\end{hscode}\resethooks

For a given number n and initial state, the function calculates
all possible n-sequences of moves that the adventures can make
\begin{hscode}\SaveRestoreHook
\column{B}{@{}>{\hspre}l<{\hspost}@{}}%
\column{E}{@{}>{\hspre}l<{\hspost}@{}}%
\>[B]{}\Varid{exec}\mathbin{::}\Conid{Int}\to \Conid{State}\to \Conid{ListDur}\;\Conid{State}{}\<[E]%
\\
\>[B]{}\Varid{exec}\mathrel{=}\bot {}\<[E]%
\ColumnHook
\end{hscode}\resethooks

Is it possible for all adventurers to be on the other side
in \ensuremath{\leq} 17 min and not exceeding 5 moves ?
\begin{hscode}\SaveRestoreHook
\column{B}{@{}>{\hspre}l<{\hspost}@{}}%
\column{E}{@{}>{\hspre}l<{\hspost}@{}}%
\>[B]{}\Varid{leq17}\mathbin{::}\Conid{Bool}{}\<[E]%
\\
\>[B]{}\Varid{leq17}\mathrel{=}\bot {}\<[E]%
\ColumnHook
\end{hscode}\resethooks

Is it possible for all adventurers to be on the other side
in $<$ 17 min ?
\begin{hscode}\SaveRestoreHook
\column{B}{@{}>{\hspre}l<{\hspost}@{}}%
\column{E}{@{}>{\hspre}l<{\hspost}@{}}%
\>[B]{}\Varid{l17}\mathbin{::}\Conid{Bool}{}\<[E]%
\\
\>[B]{}\Varid{l17}\mathrel{=}\bot {}\<[E]%
\ColumnHook
\end{hscode}\resethooks

Implementation of the monad used for the problem of the adventurers.
Recall the Knight's quest.

\begin{hscode}\SaveRestoreHook
\column{B}{@{}>{\hspre}l<{\hspost}@{}}%
\column{4}{@{}>{\hspre}l<{\hspost}@{}}%
\column{E}{@{}>{\hspre}l<{\hspost}@{}}%
\>[B]{}\mathbf{data}\;\Conid{ListDur}\;\Varid{a}\mathrel{=}\Conid{LD}\;[\mskip1.5mu \Conid{Duration}\;\Varid{a}\mskip1.5mu]\;\mathbf{deriving}\;\Conid{Show}{}\<[E]%
\\[\blanklineskip]%
\>[B]{}\Varid{remLD}\mathbin{::}\Conid{ListDur}\;\Varid{a}\to [\mskip1.5mu \Conid{Duration}\;\Varid{a}\mskip1.5mu]{}\<[E]%
\\
\>[B]{}\Varid{remLD}\;(\Conid{LD}\;\Varid{x})\mathrel{=}\Varid{x}{}\<[E]%
\\[\blanklineskip]%
\>[B]{}\mathbf{instance}\;\Conid{Functor}\;\Conid{ListDur}\;\mathbf{where}{}\<[E]%
\\
\>[B]{}\hsindent{4}{}\<[4]%
\>[4]{}\mathsf{fmap}\;\Varid{f}\mathrel{=}\bot {}\<[E]%
\\[\blanklineskip]%
\>[B]{}\mathbf{instance}\;\Conid{Applicative}\;\Conid{ListDur}\;\mathbf{where}{}\<[E]%
\\
\>[B]{}\hsindent{4}{}\<[4]%
\>[4]{}\Varid{pure}\;\Varid{x}\mathrel{=}\bot {}\<[E]%
\\
\>[B]{}\hsindent{4}{}\<[4]%
\>[4]{}l_1 \mathbin{<*>}l_2 \mathrel{=}\bot {}\<[E]%
\\[\blanklineskip]%
\>[B]{}\mathbf{instance}\;\Conid{Monad}\;\Conid{ListDur}\;\mathbf{where}{}\<[E]%
\\
\>[B]{}\hsindent{4}{}\<[4]%
\>[4]{}\Varid{return}\mathrel{=}\bot {}\<[E]%
\\
\>[B]{}\hsindent{4}{}\<[4]%
\>[4]{}\Varid{l}\bind \Varid{k}\mathrel{=}\bot {}\<[E]%
\\[\blanklineskip]%
\>[B]{}\Varid{manyChoice}\mathbin{::}[\mskip1.5mu \Conid{ListDur}\;\Varid{a}\mskip1.5mu]\to \Conid{ListDur}\;\Varid{a}{}\<[E]%
\\
\>[B]{}\Varid{manyChoice}\mathrel{=}\Conid{LD}\comp \Varid{concat}\comp (\map \;\Varid{remLD}){}\<[E]%
\ColumnHook
\end{hscode}\resethooks

%----------------- Apêndice ---------------------------------------------------%
\appendix



%----------------- Índice remissivo (exige makeindex) -------------------------%

%\printindex

%----------------- Bibliografia (exige bibtex) --------------------------------%

%\bibliographystyle{plain}
%\bibliography{tp2}

%----------------- Fim do documento -------------------------------------------%
\end{document}
