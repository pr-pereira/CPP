\documentclass[a4paper]{article}
\usepackage[a4paper,left=2cm,right=2cm,top=2cm,bottom=2cm]{geometry}
\usepackage[english]{babel}
\usepackage{palatino}
\usepackage[colorlinks=true,linkcolor=blue,citecolor=blue]{hyperref}
\usepackage{graphicx}
\usepackage{tp2}
\usepackage{subcaption}
\usepackage{adjustbox}
\usepackage{color}

\definecolor{red}{RGB}{255,  0,  0}
\definecolor{blue}{RGB}{0,0,255}
\def\red{\color{red}}
\def\blue{\color{blue}}
%================= local x=====================================================%
\def\getGif#1{\includegraphics[width=0.3\textwidth]{cp2122t_media/#1.png}}
\let\uk=\emph
\def\aspas#1{``#1"}
%================= lhs2tex=====================================================%
%% ODER: format ==         = "\mathrel{==}"
%% ODER: format /=         = "\neq "
%
%
\makeatletter
\@ifundefined{lhs2tex.lhs2tex.sty.read}%
  {\@namedef{lhs2tex.lhs2tex.sty.read}{}%
   \newcommand\SkipToFmtEnd{}%
   \newcommand\EndFmtInput{}%
   \long\def\SkipToFmtEnd#1\EndFmtInput{}%
  }\SkipToFmtEnd

\newcommand\ReadOnlyOnce[1]{\@ifundefined{#1}{\@namedef{#1}{}}\SkipToFmtEnd}
\usepackage{amstext}
\usepackage{amssymb}
\usepackage{stmaryrd}
\DeclareFontFamily{OT1}{cmtex}{}
\DeclareFontShape{OT1}{cmtex}{m}{n}
  {<5><6><7><8>cmtex8
   <9>cmtex9
   <10><10.95><12><14.4><17.28><20.74><24.88>cmtex10}{}
\DeclareFontShape{OT1}{cmtex}{m}{it}
  {<-> ssub * cmtt/m/it}{}
\newcommand{\texfamily}{\fontfamily{cmtex}\selectfont}
\DeclareFontShape{OT1}{cmtt}{bx}{n}
  {<5><6><7><8>cmtt8
   <9>cmbtt9
   <10><10.95><12><14.4><17.28><20.74><24.88>cmbtt10}{}
\DeclareFontShape{OT1}{cmtex}{bx}{n}
  {<-> ssub * cmtt/bx/n}{}
\newcommand{\tex}[1]{\text{\texfamily#1}}	% NEU

\newcommand{\Sp}{\hskip.33334em\relax}


\newcommand{\Conid}[1]{\mathit{#1}}
\newcommand{\Varid}[1]{\mathit{#1}}
\newcommand{\anonymous}{\kern0.06em \vbox{\hrule\@width.5em}}
\newcommand{\plus}{\mathbin{+\!\!\!+}}
\newcommand{\bind}{\mathbin{>\!\!\!>\mkern-6.7mu=}}
\newcommand{\rbind}{\mathbin{=\mkern-6.7mu<\!\!\!<}}% suggested by Neil Mitchell
\newcommand{\sequ}{\mathbin{>\!\!\!>}}
\renewcommand{\leq}{\leqslant}
\renewcommand{\geq}{\geqslant}
\usepackage{polytable}

%mathindent has to be defined
\@ifundefined{mathindent}%
  {\newdimen\mathindent\mathindent\leftmargini}%
  {}%

\def\resethooks{%
  \global\let\SaveRestoreHook\empty
  \global\let\ColumnHook\empty}
\newcommand*{\savecolumns}[1][default]%
  {\g@addto@macro\SaveRestoreHook{\savecolumns[#1]}}
\newcommand*{\restorecolumns}[1][default]%
  {\g@addto@macro\SaveRestoreHook{\restorecolumns[#1]}}
\newcommand*{\aligncolumn}[2]%
  {\g@addto@macro\ColumnHook{\column{#1}{#2}}}

\resethooks

\newcommand{\onelinecommentchars}{\quad-{}- }
\newcommand{\commentbeginchars}{\enskip\{-}
\newcommand{\commentendchars}{-\}\enskip}

\newcommand{\visiblecomments}{%
  \let\onelinecomment=\onelinecommentchars
  \let\commentbegin=\commentbeginchars
  \let\commentend=\commentendchars}

\newcommand{\invisiblecomments}{%
  \let\onelinecomment=\empty
  \let\commentbegin=\empty
  \let\commentend=\empty}

\visiblecomments

\newlength{\blanklineskip}
\setlength{\blanklineskip}{0.66084ex}

\newcommand{\hsindent}[1]{\quad}% default is fixed indentation
\let\hspre\empty
\let\hspost\empty
\newcommand{\NB}{\textbf{NB}}
\newcommand{\Todo}[1]{$\langle$\textbf{To do:}~#1$\rangle$}

\EndFmtInput
\makeatother
%
%
%
%
%
%
% This package provides two environments suitable to take the place
% of hscode, called "plainhscode" and "arrayhscode". 
%
% The plain environment surrounds each code block by vertical space,
% and it uses \abovedisplayskip and \belowdisplayskip to get spacing
% similar to formulas. Note that if these dimensions are changed,
% the spacing around displayed math formulas changes as well.
% All code is indented using \leftskip.
%
% Changed 19.08.2004 to reflect changes in colorcode. Should work with
% CodeGroup.sty.
%
\ReadOnlyOnce{polycode.fmt}%
\makeatletter

\newcommand{\hsnewpar}[1]%
  {{\parskip=0pt\parindent=0pt\par\vskip #1\noindent}}

% can be used, for instance, to redefine the code size, by setting the
% command to \small or something alike
\newcommand{\hscodestyle}{}

% The command \sethscode can be used to switch the code formatting
% behaviour by mapping the hscode environment in the subst directive
% to a new LaTeX environment.

\newcommand{\sethscode}[1]%
  {\expandafter\let\expandafter\hscode\csname #1\endcsname
   \expandafter\let\expandafter\endhscode\csname end#1\endcsname}

% "compatibility" mode restores the non-polycode.fmt layout.

\newenvironment{compathscode}%
  {\par\noindent
   \advance\leftskip\mathindent
   \hscodestyle
   \let\\=\@normalcr
   \let\hspre\(\let\hspost\)%
   \pboxed}%
  {\endpboxed\)%
   \par\noindent
   \ignorespacesafterend}

\newcommand{\compaths}{\sethscode{compathscode}}

% "plain" mode is the proposed default.
% It should now work with \centering.
% This required some changes. The old version
% is still available for reference as oldplainhscode.

\newenvironment{plainhscode}%
  {\hsnewpar\abovedisplayskip
   \advance\leftskip\mathindent
   \hscodestyle
   \let\hspre\(\let\hspost\)%
   \pboxed}%
  {\endpboxed%
   \hsnewpar\belowdisplayskip
   \ignorespacesafterend}

\newenvironment{oldplainhscode}%
  {\hsnewpar\abovedisplayskip
   \advance\leftskip\mathindent
   \hscodestyle
   \let\\=\@normalcr
   \(\pboxed}%
  {\endpboxed\)%
   \hsnewpar\belowdisplayskip
   \ignorespacesafterend}

% Here, we make plainhscode the default environment.

\newcommand{\plainhs}{\sethscode{plainhscode}}
\newcommand{\oldplainhs}{\sethscode{oldplainhscode}}
\plainhs

% The arrayhscode is like plain, but makes use of polytable's
% parray environment which disallows page breaks in code blocks.

\newenvironment{arrayhscode}%
  {\hsnewpar\abovedisplayskip
   \advance\leftskip\mathindent
   \hscodestyle
   \let\\=\@normalcr
   \(\parray}%
  {\endparray\)%
   \hsnewpar\belowdisplayskip
   \ignorespacesafterend}

\newcommand{\arrayhs}{\sethscode{arrayhscode}}

% The mathhscode environment also makes use of polytable's parray 
% environment. It is supposed to be used only inside math mode 
% (I used it to typeset the type rules in my thesis).

\newenvironment{mathhscode}%
  {\parray}{\endparray}

\newcommand{\mathhs}{\sethscode{mathhscode}}

% texths is similar to mathhs, but works in text mode.

\newenvironment{texthscode}%
  {\(\parray}{\endparray\)}

\newcommand{\texths}{\sethscode{texthscode}}

% The framed environment places code in a framed box.

\def\codeframewidth{\arrayrulewidth}
\RequirePackage{calc}

\newenvironment{framedhscode}%
  {\parskip=\abovedisplayskip\par\noindent
   \hscodestyle
   \arrayrulewidth=\codeframewidth
   \tabular{@{}|p{\linewidth-2\arraycolsep-2\arrayrulewidth-2pt}|@{}}%
   \hline\framedhslinecorrect\\{-1.5ex}%
   \let\endoflinesave=\\
   \let\\=\@normalcr
   \(\pboxed}%
  {\endpboxed\)%
   \framedhslinecorrect\endoflinesave{.5ex}\hline
   \endtabular
   \parskip=\belowdisplayskip\par\noindent
   \ignorespacesafterend}

\newcommand{\framedhslinecorrect}[2]%
  {#1[#2]}

\newcommand{\framedhs}{\sethscode{framedhscode}}

% The inlinehscode environment is an experimental environment
% that can be used to typeset displayed code inline.

\newenvironment{inlinehscode}%
  {\(\def\column##1##2{}%
   \let\>\undefined\let\<\undefined\let\\\undefined
   \newcommand\>[1][]{}\newcommand\<[1][]{}\newcommand\\[1][]{}%
   \def\fromto##1##2##3{##3}%
   \def\nextline{}}{\) }%

\newcommand{\inlinehs}{\sethscode{inlinehscode}}

% The joincode environment is a separate environment that
% can be used to surround and thereby connect multiple code
% blocks.

\newenvironment{joincode}%
  {\let\orighscode=\hscode
   \let\origendhscode=\endhscode
   \def\endhscode{\def\hscode{\endgroup\def\@currenvir{hscode}\\}\begingroup}
   %\let\SaveRestoreHook=\empty
   %\let\ColumnHook=\empty
   %\let\resethooks=\empty
   \orighscode\def\hscode{\endgroup\def\@currenvir{hscode}}}%
  {\origendhscode
   \global\let\hscode=\orighscode
   \global\let\endhscode=\origendhscode}%

\makeatother
\EndFmtInput
%
\def\ana#1{\mathopen{[\!(}#1\mathclose{)\!]}}
%%format (bin (n) (k)) = "\Big(\vcenter{\xymatrix@R=1pt{" n "\\" k "}}\Big)"
%%format % = "\mathbin{/}"
\newlabel{eq:fokkinga}{{3.93}{110}{The mutual-recursion law}{section.3.17}{}}
\def\mcond#1#2#3{#1 \rightarrow #2\;,\;#3}
\def\plus{\mathbin{\dagger}}

%---------------------------------------------------------------------------

\title{\bfseries Modelling and Analysis of a Cyber-Physical System\\ with Monads \\ {\Large Cyber-Physical Programming --- Practical Assignment 2}}

\author{
    Melânia Pereira \quad \quad Paulo R. Pereira\\
    \texttt{\{pg47520, pg47554\}@alunos.uminho.pt}
}

\begin{document}
\raggedbottom
\setstretch{1.25}

\maketitle

\begin{abstract}
ola
\end{abstract}


\section{The Adventurers' Problem}
In the middle of the night, four adventurers encounter a shabby rope-bridge spanning a deep ravine.
For safety reasons, they decide that no more than 2 people should cross the bridge at the same
time and that a flashlight needs to be carried by one of them in every crossing. They have only
one flashlight. The 4 adventurers are not equally skilled: crossing the bridge takes them 1, 2, 5,
and 10 minutes, respectively. A pair of adventurers crosses the bridge in an amount of time equal
to that of the slowest of the two adventurers.

One of the adventurers claims that they cannot be all on the other side in less than 19 minutes.
One companion disagrees and claims that it can be done in 17 minutes.

Who is right? That's what we're going to find out.

\section{Monadic Approach via \textsc{Haskell} for Modelling the Problem}
\subsection{The monads used}
\fbox{explain the monads here}
\subsection{Modelling the problem}
Adventurers are represented by the following data type:
\begin{hscode}\SaveRestoreHook
\column{B}{@{}>{\hspre}l<{\hspost}@{}}%
\column{E}{@{}>{\hspre}l<{\hspost}@{}}%
\>[B]{}\mathbf{data}\;\Conid{Adventurer}\mathrel{=}P_1 \mid P_2 \mid P_5 \mid P_{10} \;\mathbf{deriving}\;(\Conid{Show},\Conid{Eq}){}\<[E]%
\ColumnHook
\end{hscode}\resethooks
Lantern is represented by the \ensuremath{()} element, so we can represent all the entities by using the coproduct and defining the following data type:
\begin{hscode}\SaveRestoreHook
\column{B}{@{}>{\hspre}l<{\hspost}@{}}%
\column{E}{@{}>{\hspre}l<{\hspost}@{}}%
\>[B]{}\mathbf{type}\;\Conid{Object}\mathrel{=}\Conid{Adventurer}+(){}\<[E]%
\\[\blanklineskip]%
\>[B]{}\Varid{lantern}\mathrel{=}i_2\;(){}\<[E]%
\ColumnHook
\end{hscode}\resethooks
The names for the adventurers are quite suggestive as they are identified by the time they take to cross. However, it will be very useful to have a function that returns, for each adventurer, the time it takes to cross the bridge.
\begin{hscode}\SaveRestoreHook
\column{B}{@{}>{\hspre}l<{\hspost}@{}}%
\column{E}{@{}>{\hspre}l<{\hspost}@{}}%
\>[B]{}\Varid{getTimeAdv}\mathbin{::}\Conid{Adventurer}\to \Conid{Int}{}\<[E]%
\\
\>[B]{}\Varid{getTimeAdv}\;P_1 \mathrel{=}\mathrm{1}{}\<[E]%
\\
\>[B]{}\Varid{getTimeAdv}\;P_2 \mathrel{=}\mathrm{2}{}\<[E]%
\\
\>[B]{}\Varid{getTimeAdv}\;P_5 \mathrel{=}\mathrm{5}{}\<[E]%
\\
\>[B]{}\Varid{getTimeAdv}\;P_{10} \mathrel{=}\mathrm{10}{}\<[E]%
\ColumnHook
\end{hscode}\resethooks
Now, we need to define the state of the game, i.e. the current position of each object (adventurers $+$ the lantern). The function \ensuremath{\underline{\Conid{False}}} represents the initial state
of the game, with all adventurers and the lantern on the left side of
the bridge. Similarly, the function \ensuremath{\underline{\Conid{True}}} represents the end
state of the game, with all adventurers and the lantern on the right
side of the bridge. We also need to define the instances \ensuremath{\Conid{Show}} and \ensuremath{\Conid{Eq}} to visualize and compare, respectively, the states of the game.
\begin{hscode}\SaveRestoreHook
\column{B}{@{}>{\hspre}l<{\hspost}@{}}%
\column{3}{@{}>{\hspre}l<{\hspost}@{}}%
\column{14}{@{}>{\hspre}l<{\hspost}@{}}%
\column{21}{@{}>{\hspre}l<{\hspost}@{}}%
\column{27}{@{}>{\hspre}l<{\hspost}@{}}%
\column{E}{@{}>{\hspre}l<{\hspost}@{}}%
\>[B]{}\mathbf{type}\;\Conid{State}\mathrel{=}\Conid{Object}\to \Conid{Bool}{}\<[E]%
\\[\blanklineskip]%
\>[B]{}\mathbf{instance}\;\Conid{Show}\;\Conid{State}\;\mathbf{where}{}\<[E]%
\\
\>[B]{}\hsindent{3}{}\<[3]%
\>[3]{}\Varid{show}\;\Varid{s}\mathrel{=}\Varid{show}\comp \Varid{show}\mathbin{\$}[\mskip1.5mu \Varid{s}\;(i_1\;P_1 ),{}\<[E]%
\\
\>[3]{}\hsindent{24}{}\<[27]%
\>[27]{}\Varid{s}\;(i_1\;P_2 ),{}\<[E]%
\\
\>[3]{}\hsindent{24}{}\<[27]%
\>[27]{}\Varid{s}\;(i_1\;P_5 ),{}\<[E]%
\\
\>[3]{}\hsindent{24}{}\<[27]%
\>[27]{}\Varid{s}\;(i_1\;P_{10} ),{}\<[E]%
\\
\>[3]{}\hsindent{24}{}\<[27]%
\>[27]{}\Varid{s}\;(i_2\;())\mskip1.5mu]{}\<[E]%
\\[\blanklineskip]%
\>[B]{}\mathbf{instance}\;\Conid{Eq}\;\Conid{State}\;\mathbf{where}{}\<[E]%
\\
\>[B]{}\hsindent{3}{}\<[3]%
\>[3]{}(\equiv )\;\Varid{s1}\;\Varid{s2}\mathrel{=}\Varid{and}\;[\mskip1.5mu \Varid{s1}\;(i_1\;P_1 )\equiv \Varid{s2}\;(i_1\;P_1 ),{}\<[E]%
\\
\>[3]{}\hsindent{18}{}\<[21]%
\>[21]{}\Varid{s1}\;(i_1\;P_2 )\equiv \Varid{s2}\;(i_1\;P_2 ),{}\<[E]%
\\
\>[3]{}\hsindent{18}{}\<[21]%
\>[21]{}\Varid{s1}\;(i_1\;P_5 )\equiv \Varid{s2}\;(i_1\;P_5 ),{}\<[E]%
\\
\>[3]{}\hsindent{18}{}\<[21]%
\>[21]{}\Varid{s1}\;(i_1\;P_{10} )\equiv \Varid{s2}\;(i_1\;P_{10} ),{}\<[E]%
\\
\>[3]{}\hsindent{18}{}\<[21]%
\>[21]{}\Varid{s1}\;(i_2\;())\equiv \Varid{s2}\;(i_2\;())\mskip1.5mu]{}\<[E]%
\\[\blanklineskip]%
\>[B]{}\Varid{gInit}\mathbin{::}\Conid{State}{}\<[E]%
\\
\>[B]{}\Varid{gInit}\mathrel{=}\underline{\Conid{False}}{}\<[E]%
\\[\blanklineskip]%
\>[B]{}\Varid{gEnd}\mathbin{::}\Conid{State}{}\<[E]%
\\
\>[B]{}\Varid{gEnd}\mathrel{=}\underline{\Conid{True}}{}\<[E]%
\\[\blanklineskip]%
\>[B]{}\Varid{state2List}\mathbin{::}\Conid{State}\to [\mskip1.5mu \Conid{Bool}\mskip1.5mu]{}\<[E]%
\\
\>[B]{}\Varid{state2List}\;\Varid{s}\mathrel{=}[\mskip1.5mu \Varid{s}\;(i_1\;P_1 ),{}\<[E]%
\\
\>[B]{}\hsindent{14}{}\<[14]%
\>[14]{}\Varid{s}\;(i_1\;P_2 ),{}\<[E]%
\\
\>[B]{}\hsindent{14}{}\<[14]%
\>[14]{}\Varid{s}\;(i_1\;P_5 ),{}\<[E]%
\\
\>[B]{}\hsindent{14}{}\<[14]%
\>[14]{}\Varid{s}\;(i_1\;P_{10} ),{}\<[E]%
\\
\>[B]{}\hsindent{14}{}\<[14]%
\>[14]{}\Varid{s}\;(i_2\;())\mskip1.5mu]{}\<[E]%
\ColumnHook
\end{hscode}\resethooks
Changes the state of the game for a given object:
\begin{hscode}\SaveRestoreHook
\column{B}{@{}>{\hspre}l<{\hspost}@{}}%
\column{E}{@{}>{\hspre}l<{\hspost}@{}}%
\>[B]{}\Varid{changeState}\mathbin{::}\Conid{Object}\to \Conid{State}\to \Conid{State}{}\<[E]%
\\
\>[B]{}\Varid{changeState}\;\Varid{a}\;\Varid{s}\mathrel{=}\mathbf{let}\;\Varid{v}\mathrel{=}\Varid{s}\;\Varid{a}\;\mathbf{in}\;(\lambda \Varid{x}\to \mathbf{if}\;\Varid{x}\equiv \Varid{a}\;\mathbf{then}\;\neg \;\Varid{v}\;\mathbf{else}\;\Varid{s}\;\Varid{x}){}\<[E]%
\ColumnHook
\end{hscode}\resethooks
Changes the state of the game of a list of Object 
\begin{hscode}\SaveRestoreHook
\column{B}{@{}>{\hspre}l<{\hspost}@{}}%
\column{E}{@{}>{\hspre}l<{\hspost}@{}}%
\>[B]{}\Varid{mChangeState}\mathbin{::}[\mskip1.5mu \Conid{Object}\mskip1.5mu]\to \Conid{State}\to \Conid{State}{}\<[E]%
\\
\>[B]{}\Varid{mChangeState}\;\Varid{os}\;\Varid{s}\mathrel{=}\Varid{foldr}\;\Varid{changeState}\;\Varid{s}\;\Varid{os}{}\<[E]%
\ColumnHook
\end{hscode}\resethooks

For a given state of the game, the function presents all the
possible moves that the adventurers can make.
\begin{hscode}\SaveRestoreHook
\column{B}{@{}>{\hspre}l<{\hspost}@{}}%
\column{3}{@{}>{\hspre}l<{\hspost}@{}}%
\column{7}{@{}>{\hspre}l<{\hspost}@{}}%
\column{E}{@{}>{\hspre}l<{\hspost}@{}}%
\>[B]{}\Varid{allValidPlays}\mathbin{::}\Conid{State}\to \Conid{ListLogDur}\;\Conid{State}{}\<[E]%
\\
\>[B]{}\Varid{allValidPlays}\;\Varid{s}\mathrel{=}\Conid{LSD}\mathbin{\$}\map \;\Conid{Duration}\mathbin{\$}\map \;(\Varid{id}\times\conj{\Varid{toTrace}\;\Varid{s}}{\Varid{id}}\comp (\Varid{mCS}\;\Varid{s}))\;\Varid{t}\;\mathbf{where}{}\<[E]%
\\
\>[B]{}\hsindent{3}{}\<[3]%
\>[3]{}\Varid{t}\mathrel{=}(\map \;(\Varid{addLantern}\comp \Varid{addTime})\comp \Varid{combinationsUpTo2}\comp \Varid{advsWhereLanternIs})\;\Varid{s}{}\<[E]%
\\
\>[B]{}\hsindent{3}{}\<[3]%
\>[3]{}\Varid{mCS}\mathrel{=}\Varid{flip}\;\Varid{mChangeState}{}\<[E]%
\\
\>[B]{}\hsindent{3}{}\<[3]%
\>[3]{}\Varid{toTrace}\;\Varid{s}\;\Varid{s'}\mathrel{=}\Varid{printTrace}\;(\Varid{state2List}\;\Varid{s},\Varid{state2List}\;\Varid{s'}){}\<[E]%
\\[\blanklineskip]%
\>[B]{}\Varid{addTime}\mathbin{::}[\mskip1.5mu \Conid{Adventurer}\mskip1.5mu]\to (\Conid{Int},[\mskip1.5mu \Conid{Adventurer}\mskip1.5mu]){}\<[E]%
\\
\>[B]{}\Varid{addTime}\mathrel{=}\conj{\Varid{maximum}\comp (\map \;\Varid{getTimeAdv})}{\Varid{id}}{}\<[E]%
\\[\blanklineskip]%
\>[B]{}\Varid{addLantern}\mathbin{::}(\Conid{Int},[\mskip1.5mu \Conid{Adventurer}\mskip1.5mu])\to (\Conid{Int},[\mskip1.5mu \Conid{Object}\mskip1.5mu]){}\<[E]%
\\
\>[B]{}\Varid{addLantern}\mathrel{=}\Varid{id}\times((\Varid{lantern}\mathbin{:})\comp \map \;i_1){}\<[E]%
\\[\blanklineskip]%
\>[B]{}\Varid{advsWhereLanternIs}\mathbin{::}\Conid{State}\to [\mskip1.5mu \Conid{Adventurer}\mskip1.5mu]{}\<[E]%
\\
\>[B]{}\Varid{advsWhereLanternIs}\;\Varid{s}\mathrel{=}\Varid{filter}\;((\equiv \Varid{s}\;\Varid{lantern})\comp \Varid{s}\comp i_1)\;[\mskip1.5mu P_1 ,P_2 ,P_5 ,P_{10} \mskip1.5mu]{}\<[E]%
\\[\blanklineskip]%
\>[B]{}\Varid{combinationsUpTo2}\mathbin{::}\Conid{Eq}\;\Varid{a}\Rightarrow [\mskip1.5mu \Varid{a}\mskip1.5mu]\to [\mskip1.5mu [\mskip1.5mu \Varid{a}\mskip1.5mu]\mskip1.5mu]{}\<[E]%
\\
\>[B]{}\Varid{combinationsUpTo2}\mathrel{=}\mathsf{conc}\comp \conj{\Varid{f}}{\Varid{g}}\;\mathbf{where}{}\<[E]%
\\
\>[B]{}\hsindent{7}{}\<[7]%
\>[7]{}\Varid{f}\;\Varid{t}\mathrel{=}\mathbf{do}\;\{\mskip1.5mu \Varid{x}\leftarrow \Varid{t};\Varid{return}\;[\mskip1.5mu \Varid{x}\mskip1.5mu]\mskip1.5mu\}{}\<[E]%
\\
\>[B]{}\hsindent{7}{}\<[7]%
\>[7]{}\Varid{g}\;\Varid{t}\mathrel{=}\mathbf{do}\;\{\mskip1.5mu \Varid{x}\leftarrow \Varid{t};\Varid{y}\leftarrow (\Varid{remove}\;\Varid{x}\;\Varid{t});\Varid{return}\;[\mskip1.5mu \Varid{x},\Varid{y}\mskip1.5mu]\mskip1.5mu\}{}\<[E]%
\\
\>[B]{}\hsindent{7}{}\<[7]%
\>[7]{}\Varid{remove}\;\Varid{x}\;[\mskip1.5mu \mskip1.5mu]\mathrel{=}[\mskip1.5mu \mskip1.5mu]{}\<[E]%
\\
\>[B]{}\hsindent{7}{}\<[7]%
\>[7]{}\Varid{remove}\;\Varid{x}\;(\Varid{h}\mathbin{:}\Varid{t})\mathrel{=}\mathbf{if}\;\Varid{x}\equiv \Varid{h}\;\mathbf{then}\;\Varid{t}\;\mathbf{else}\;\Varid{remove}\;\Varid{x}\;\Varid{t}{}\<[E]%
\ColumnHook
\end{hscode}\resethooks
\begin{tabbing}\ttfamily
~~\char62{}~combinationsUpTo2~\char91{}1\char44{}2\char44{}3\char93{}\\
\ttfamily ~~\char91{}\char91{}1\char93{}\char44{}~\char91{}2\char93{}\char44{}~\char91{}3\char93{}\char44{}~\char91{}1\char44{}2\char93{}\char44{}~\char91{}1\char44{}3\char93{}\char44{}~\char91{}2\char44{}3\char93{}\char93{}
\end{tabbing}

\subsubsection{The trace log}
As we saw, our monad \ensuremath{\Conid{ListLogDur}} keeps the trace by calling the function \ensuremath{\Varid{toTrace}\mathbin{::}\Conid{State}\to \Conid{State}\to \Conid{String}}. But what does it do?

First, we can see that, according to the representation of the state, adventurers can be represented by indexes. We take advantage of this to be able to present an elegant trace of the moves. For example, if the previous state is \ensuremath{[\mskip1.5mu \Conid{False},\Conid{False},\Conid{False},\Conid{False},\Conid{False}\mskip1.5mu]} and the current state is \ensuremath{[\mskip1.5mu \Conid{True},\Conid{True},\Conid{False},\Conid{False},\Conid{True}\mskip1.5mu]}, we know that \ensuremath{P_1 } and \ensuremath{P_2 } have crossed (because the first two and the last elements and diferent). So, we can simply compare element to element and, if they are different, we keep the index. In the previous example, it would return \ensuremath{[\mskip1.5mu \mathrm{0},\mathrm{1},\mathrm{4}\mskip1.5mu]} --- index 4 represents the lantern, and because we assume that the movements are always valid, we can ignore that.
\begin{hscode}\SaveRestoreHook
\column{B}{@{}>{\hspre}l<{\hspost}@{}}%
\column{3}{@{}>{\hspre}l<{\hspost}@{}}%
\column{30}{@{}>{\hspre}l<{\hspost}@{}}%
\column{E}{@{}>{\hspre}l<{\hspost}@{}}%
\>[B]{}\Varid{index2Adv}\mathbin{::}\Conid{Int}\to \Conid{String}{}\<[E]%
\\
\>[B]{}\Varid{index2Adv}\;\mathrm{0}\mathrel{=}\text{\ttfamily \char34 P1\char34}{}\<[E]%
\\
\>[B]{}\Varid{index2Adv}\;\mathrm{1}\mathrel{=}\text{\ttfamily \char34 P2\char34}{}\<[E]%
\\
\>[B]{}\Varid{index2Adv}\;\mathrm{2}\mathrel{=}\text{\ttfamily \char34 P5\char34}{}\<[E]%
\\
\>[B]{}\Varid{index2Adv}\;\mathrm{3}\mathrel{=}\text{\ttfamily \char34 P10\char34}{}\<[E]%
\\[\blanklineskip]%
\>[B]{}\Varid{indexesWithDifferentValues}\mathbin{::}\Conid{Eq}\;\Varid{a}\Rightarrow ([\mskip1.5mu \Varid{a}\mskip1.5mu],[\mskip1.5mu \Varid{a}\mskip1.5mu])\to [\mskip1.5mu \Conid{Int}\mskip1.5mu]{}\<[E]%
\\
\>[B]{}\Varid{indexesWithDifferentValues}\;(l_1 ,l_2 )\mathrel{=}\Varid{aux}\;l_1 \;l_2 \;\mathrm{0}\;\mathbf{where}{}\<[E]%
\\
\>[B]{}\hsindent{3}{}\<[3]%
\>[3]{}\Varid{aux}\mathbin{::}\Conid{Eq}\;\Varid{a}\Rightarrow [\mskip1.5mu \Varid{a}\mskip1.5mu]\to [\mskip1.5mu \Varid{a}\mskip1.5mu]\to \Conid{Int}\to [\mskip1.5mu \Conid{Int}\mskip1.5mu]{}\<[E]%
\\
\>[B]{}\hsindent{3}{}\<[3]%
\>[3]{}\Varid{aux}\;[\mskip1.5mu \mskip1.5mu]\;\Varid{l}\;\anonymous \mathrel{=}[\mskip1.5mu \mskip1.5mu]{}\<[E]%
\\
\>[B]{}\hsindent{3}{}\<[3]%
\>[3]{}\Varid{aux}\;\Varid{l}\;[\mskip1.5mu \mskip1.5mu]\;\anonymous \mathrel{=}[\mskip1.5mu \mskip1.5mu]{}\<[E]%
\\
\>[B]{}\hsindent{3}{}\<[3]%
\>[3]{}\Varid{aux}\;(h_1 \mathbin{:}\Varid{t1})\;(h_2 \mathbin{:}\Varid{t2})\;\Varid{index}\mathrel{=}\mathbf{if}\;h_1 \not\equiv h_2 \;\mathbf{then}\;\Varid{index}\mathbin{:}\Varid{aux}\;\Varid{t1}\;\Varid{t2}\;(\Varid{index}\mathbin{+}\mathrm{1}){}\<[E]%
\\
\>[3]{}\hsindent{27}{}\<[30]%
\>[30]{}\mathbf{else}\;\Varid{aux}\;\Varid{t1}\;\Varid{t2}\;(\Varid{index}\mathbin{+}\mathrm{1}){}\<[E]%
\ColumnHook
\end{hscode}\resethooks
The result \ensuremath{[\mskip1.5mu \mathrm{0},\mathrm{1},\mathrm{4}\mskip1.5mu]} means that \aspas{\ensuremath{P_1 } and \ensuremath{P_2 } crosses}. We now have automate this (pretty) print. We only need to ignore the lantern index (4), convert the indexes to the respective adventurers and define a print function for them.
\begin{hscode}\SaveRestoreHook
\column{B}{@{}>{\hspre}l<{\hspost}@{}}%
\column{5}{@{}>{\hspre}l<{\hspost}@{}}%
\column{E}{@{}>{\hspre}l<{\hspost}@{}}%
\>[B]{}\Varid{printTrace}\mathbin{::}([\mskip1.5mu \Conid{Bool}\mskip1.5mu],[\mskip1.5mu \Conid{Bool}\mskip1.5mu])\to \Conid{String}{}\<[E]%
\\
\>[B]{}\Varid{printTrace}\mathrel{=}\Varid{prettyLog}\comp (\map \;\Varid{index2Adv})\comp \Varid{init}\comp \Varid{indexesWithDifferentValues}{}\<[E]%
\\[\blanklineskip]%
\>[B]{}\Varid{prettyLog}\mathbin{::}[\mskip1.5mu \Conid{String}\mskip1.5mu]\to \Conid{String}{}\<[E]%
\\
\>[B]{}\Varid{prettyLog}\mathrel{=}\mcond{(\mathbin{>}\mathrm{1})\comp \length }{\Varid{f}}{(\mathbin{+\!\!\!+}\text{\ttfamily \char34 ~cross\char92 n\char34})\comp \Varid{head}}\;\mathbf{where}{}\<[E]%
\\
\>[B]{}\hsindent{5}{}\<[5]%
\>[5]{}\Varid{f}\mathrel{=}(\mathbin{+\!\!\!+}\text{\ttfamily \char34 ~crosses\char92 n\char34})\comp \mathsf{conc}\comp ((\Varid{concat}\comp \map \;(\mathbin{+\!\!\!+}\text{\ttfamily \char34 ~and~\char34}))\times\Varid{id})\comp \conj{\Varid{init}}{\Varid{last}}{}\<[E]%
\ColumnHook
\end{hscode}\resethooks
Let's see the result of applying the function \ensuremath{\Varid{printTrace}} with the previous example.
\begin{tabbing}\ttfamily
~~\char62{}~t~\char61{}~\char40{}\char91{}False\char44{}False\char44{}False\char44{}False\char44{}False\char93{}\char44{}\char91{}True\char44{}True\char44{}False\char44{}False\char44{}True\char93{}\char41{}\\
\ttfamily ~~\char62{}~printTrace~t\\
\ttfamily ~~\char34{}P1~and~P2~crosses\char92{}n\char34{}
\end{tabbing}
Finnaly, using the function \ensuremath{\Varid{putStr}}, we get a pretty nice log:
\begin{tabbing}\ttfamily
~~\char62{}~putStr~\char36{}~printTrace~t\\
\ttfamily ~~P1~and~P2~crosses
\end{tabbing}
In the next subsection, we'll see the trace of the optimal play which shows how elegant the log is. 
\subsection{Solving the problem}
For a given number n and initial state, the function calculates
all possible n-sequences of moves that the adventures can make
\begin{hscode}\SaveRestoreHook
\column{B}{@{}>{\hspre}l<{\hspost}@{}}%
\column{15}{@{}>{\hspre}l<{\hspost}@{}}%
\column{16}{@{}>{\hspre}l<{\hspost}@{}}%
\column{17}{@{}>{\hspre}l<{\hspost}@{}}%
\column{19}{@{}>{\hspre}l<{\hspost}@{}}%
\column{33}{@{}>{\hspre}l<{\hspost}@{}}%
\column{35}{@{}>{\hspre}l<{\hspost}@{}}%
\column{E}{@{}>{\hspre}l<{\hspost}@{}}%
\>[B]{}\Varid{exec}\mathbin{::}\Conid{Int}\to \Conid{State}\to \Conid{ListLogDur}\;\Conid{State}{}\<[E]%
\\
\>[B]{}\Varid{exec}\;\mathrm{0}\;\Varid{s}\mathrel{=}\Varid{allValidPlays}\;\Varid{s}{}\<[E]%
\\
\>[B]{}\Varid{exec}\;\Varid{n}\;\Varid{s}\mathrel{=}\mathbf{do}\;\Varid{ps}\leftarrow \Varid{exec}\;(\Varid{n}\mathbin{-}\mathrm{1})\;\Varid{s}{}\<[E]%
\\
\>[B]{}\hsindent{15}{}\<[15]%
\>[15]{}\Varid{allValidPlays}\;\Varid{ps}{}\<[E]%
\\[\blanklineskip]%
\>[B]{}\Varid{execPred}\mathbin{::}(\Conid{State}\to \Conid{Bool})\to \Conid{State}\to (\Conid{Int},\Conid{ListLogDur}\;\Conid{State}){}\<[E]%
\\
\>[B]{}\Varid{execPred}\;\Varid{p}\;\Varid{s}\mathrel{=}\Varid{aux}\;\Varid{p}\;\Varid{s}\;\mathrm{0}\;\mathbf{where}{}\<[E]%
\\
\>[B]{}\hsindent{16}{}\<[16]%
\>[16]{}\Varid{aux}\;\Varid{p}\;\Varid{s}\;\Varid{it}\mathrel{=}\mathbf{let}\;\Varid{st}\mathrel{=}\Varid{exec}\;\Varid{it}\;\Varid{s}{}\<[E]%
\\
\>[16]{}\hsindent{17}{}\<[33]%
\>[33]{}\Varid{res}\mathrel{=}\Varid{filter}\;\Varid{pred}\;(\map \;\Varid{remDur}\;(\Varid{remLSD}\;\Varid{st}))\;\mathbf{in}{}\<[E]%
\\
\>[16]{}\hsindent{17}{}\<[33]%
\>[33]{}\mathbf{if}\;\length \;(\Varid{res})\mathbin{>}\mathrm{0}\;\mathbf{then}\;((\Varid{it}\mathbin{+}\mathrm{1}),\Conid{LSD}\;(\map \;\Conid{Duration}\;\Varid{res})){}\<[E]%
\\
\>[16]{}\hsindent{17}{}\<[33]%
\>[33]{}\mathbf{else}\;\Varid{aux}\;\Varid{p}\;\Varid{s}\;(\Varid{it}\mathbin{+}\mathrm{1})\;\mathbf{where}{}\<[E]%
\\
\>[33]{}\hsindent{2}{}\<[35]%
\>[35]{}\Varid{remDur}\;(\Conid{Duration}\;\Varid{a})\mathrel{=}\Varid{a}{}\<[E]%
\\
\>[33]{}\hsindent{2}{}\<[35]%
\>[35]{}\Varid{pred}\;(\anonymous ,(\anonymous ,\Varid{s}))\mathrel{=}\Varid{p}\;\Varid{s}{}\<[E]%
\\[\blanklineskip]%
\>[B]{}\Varid{leqX}\mathbin{::}\Conid{Int}\to (\Conid{Int},\Conid{Bool}){}\<[E]%
\\
\>[B]{}\Varid{leqX}\;\Varid{n}\mathrel{=}\mathbf{if}\;\Varid{res}\;\mathbf{then}\;(\Varid{it},\Varid{res}){}\<[E]%
\\
\>[B]{}\hsindent{17}{}\<[17]%
\>[17]{}\mathbf{else}\;(\mathrm{0},\Varid{res})\;\mathbf{where}{}\<[E]%
\\
\>[17]{}\hsindent{2}{}\<[19]%
\>[19]{}\Varid{res}\mathrel{=}\length \;(\Varid{filter}\;\Varid{p}\;(\map \;\Varid{remDur}\;(\Varid{remLSD}\;\Varid{l})))\mathbin{>}\mathrm{0}{}\<[E]%
\\
\>[17]{}\hsindent{2}{}\<[19]%
\>[19]{}(\Varid{it},\Varid{l})\mathrel{=}\Varid{execPred}\;(\equiv \Varid{gEnd})\;\Varid{gInit}{}\<[E]%
\\
\>[17]{}\hsindent{2}{}\<[19]%
\>[19]{}\Varid{p}\;(\Varid{d},(\anonymous ,\anonymous ))\mathrel{=}\Varid{d}\leq \Varid{n}{}\<[E]%
\\
\>[17]{}\hsindent{2}{}\<[19]%
\>[19]{}\Varid{remDur}\;(\Conid{Duration}\;\Varid{a})\mathrel{=}\Varid{a}{}\<[E]%
\\[\blanklineskip]%
\>[B]{}\Varid{lX}\mathbin{::}\Conid{Int}\to (\Conid{Int},\Conid{Bool}){}\<[E]%
\\
\>[B]{}\Varid{lX}\;\Varid{n}\mathrel{=}\mathbf{if}\;\Varid{res}\;\mathbf{then}\;(\Varid{it},\Varid{res}){}\<[E]%
\\
\>[B]{}\hsindent{15}{}\<[15]%
\>[15]{}\mathbf{else}\;(\mathrm{0},\Varid{res})\;\mathbf{where}{}\<[E]%
\\
\>[15]{}\hsindent{2}{}\<[17]%
\>[17]{}\Varid{res}\mathrel{=}\length \;(\Varid{filter}\;\Varid{p}\;(\map \;\Varid{remDur}\;(\Varid{remLSD}\;\Varid{l})))\mathbin{>}\mathrm{0}{}\<[E]%
\\
\>[15]{}\hsindent{2}{}\<[17]%
\>[17]{}(\Varid{it},\Varid{l})\mathrel{=}\Varid{execPred}\;(\equiv \Varid{gEnd})\;\Varid{gInit}{}\<[E]%
\\
\>[15]{}\hsindent{2}{}\<[17]%
\>[17]{}\Varid{p}\;(\Varid{d},(\anonymous ,\anonymous ))\mathrel{=}\Varid{d}\mathbin{<}\Varid{n}{}\<[E]%
\\
\>[15]{}\hsindent{2}{}\<[17]%
\>[17]{}\Varid{remDur}\;(\Conid{Duration}\;\Varid{a})\mathrel{=}\Varid{a}{}\<[E]%
\ColumnHook
\end{hscode}\resethooks

\textbf{Question}: Is it possible for all adventurers to be on the other side
in \ensuremath{\leq \mathrm{17}} minutes and not exceeding 5 moves?
\begin{hscode}\SaveRestoreHook
\column{B}{@{}>{\hspre}l<{\hspost}@{}}%
\column{E}{@{}>{\hspre}l<{\hspost}@{}}%
\>[B]{}\Varid{leq17}\mathbin{::}\Conid{Bool}{}\<[E]%
\\
\>[B]{}\Varid{leq17}\mathrel{=}\p2\;(\Varid{leqX}\;\mathrm{17})\mathrel{\wedge}\p1\;(\Varid{leqX}\;\mathrm{17})\leq \mathrm{5}{}\<[E]%
\ColumnHook
\end{hscode}\resethooks

\textbf{Question}: Is it possible for all adventurers to be on the other side
in \ensuremath{\mathbin{<}\mathrm{17}} minutes?
\begin{hscode}\SaveRestoreHook
\column{B}{@{}>{\hspre}l<{\hspost}@{}}%
\column{E}{@{}>{\hspre}l<{\hspost}@{}}%
\>[B]{}\Varid{l17}\mathbin{::}\Conid{Bool}{}\<[E]%
\\
\>[B]{}\Varid{l17}\mathrel{=}\p2\;(\Varid{lX}\;\mathrm{17}){}\<[E]%
\ColumnHook
\end{hscode}\resethooks
As we saw, it is possible for all adventurers to be on the other side
in \ensuremath{\leq \mathrm{17}} minutes and not exceeding 5 moves (actually we exactly 5 moves). We also prooved that it isn't possible for all adventurers to be on the other side in \ensuremath{\mathbin{<}\mathrm{17}} minutes. So, one could get that information by executing the following function \textit{optimalTrace}.
\begin{hscode}\SaveRestoreHook
\column{B}{@{}>{\hspre}l<{\hspost}@{}}%
\column{9}{@{}>{\hspre}l<{\hspost}@{}}%
\column{E}{@{}>{\hspre}l<{\hspost}@{}}%
\>[B]{}\Varid{optimalTrace}\mathbin{::}\fun{IO}\;(){}\<[E]%
\\
\>[B]{}\Varid{optimalTrace}\mathrel{=}{}\<[E]%
\\
\>[B]{}\hsindent{9}{}\<[9]%
\>[9]{}\Varid{putStrLn}\comp \Varid{t}\comp \map \;\Varid{remDur}\comp \Varid{remLSD}\comp \p2\mathbin{\$}\Varid{execPred}\;(\equiv \Varid{gEnd})\;\Varid{gInit}\;\mathbf{where}{}\<[E]%
\\
\>[B]{}\hsindent{9}{}\<[9]%
\>[9]{}\Varid{t}\mathrel{=}\Varid{prt}\comp \conj{\Varid{head}\comp \map \;\p1}{\map \;(\p1\comp \p2)}\comp \Varid{pairFilter}\comp \conj{\Varid{minimum}\comp \map \;\p1}{\Varid{id}}{}\<[E]%
\\
\>[B]{}\hsindent{9}{}\<[9]%
\>[9]{}\Varid{remDur}\;(\Conid{Duration}\;\Varid{a})\mathrel{=}\Varid{a}{}\<[E]%
\\
\>[B]{}\hsindent{9}{}\<[9]%
\>[9]{}\Varid{pairFilter}\;(\Varid{d},\Varid{l})\mathrel{=}\Varid{filter}\;(\lambda (\Varid{d'},(\anonymous ,\anonymous ))\to \Varid{d}\equiv \Varid{d'})\;\Varid{l}{}\<[E]%
\\
\>[B]{}\hsindent{9}{}\<[9]%
\>[9]{}\Varid{p}\mathrel{=}\mcond{(\mathbin{>}\mathrm{1})\comp \length }{\Varid{p'}}{\Varid{head}}{}\<[E]%
\\
\>[B]{}\hsindent{9}{}\<[9]%
\>[9]{}\Varid{p'}\mathrel{=}\mathsf{conc}\comp \conj{\Varid{concat}\comp \map \;((\mathbin{+\!\!\!+}(\text{\ttfamily \char34 \char92 nOR\char92 n\char92 n\char34})))\comp \Varid{init}}{\Varid{last}}{}\<[E]%
\\
\>[B]{}\hsindent{9}{}\<[9]%
\>[9]{}\Varid{prt}\;(\Varid{d},\Varid{l})\mathrel{=}(\Varid{p}\;\Varid{l})\mathbin{+\!\!\!+}\text{\ttfamily \char34 \char92 nin~\char34}\mathbin{+\!\!\!+}(\Varid{show}\;\Varid{d})\mathbin{+\!\!\!+}\text{\ttfamily \char34 ~minutes.\char34}{}\<[E]%
\ColumnHook
\end{hscode}\resethooks
Result:
\begin{tabbing}\ttfamily
~~\char62{}~optimalTrace~\\
\ttfamily ~~P1~and~P2~crosses\\
\ttfamily ~~P1~cross\\
\ttfamily ~~P5~and~P10~crosses\\
\ttfamily ~~P2~cross\\
\ttfamily ~~P1~and~P2~crosses\\
\ttfamily ~~\\
\ttfamily ~~OR\\
\ttfamily ~~\\
\ttfamily ~~P1~and~P2~crosses\\
\ttfamily ~~P2~cross\\
\ttfamily ~~P5~and~P10~crosses\\
\ttfamily ~~P1~cross\\
\ttfamily ~~P1~and~P2~crosses\\
\ttfamily ~~\\
\ttfamily ~~in~17~minutes\char46{}
\end{tabbing}
\section{Comparative Analysis and Final Comments}

\end{document}
